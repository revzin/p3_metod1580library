% -*-coding: utf-8 -*-
% Держать в начале каждого файла!

\documentclass[russian, 12pt]{article}
\usepackage{metod}

\MTDSetPhysSection{Механика}
\MTDSetTitle{Введение в теорию погрешностей. Измерения в физическом практикуме}
\MTDDesignator{М0}
\MTDSetGrade{10}

\MTDSetAuthors{И.~Н.~Грачева, В.~И.~Гребенкин, А.~Е.~Иванов, И.~А.~Коротова,
Е.~И.~Красавина, А.~В.~Кравцов, Н.~С.~Кулеба, Б.~В.~Падалкин,
Г.~Ю.~Шевцова, Т.~С.~Цвецинская}

\MTDSetEditorsGenCase{И. Н. Грачевой, А. Е. Иванова, А. В. Кравцова}

\begin{document}

\MTDTitlePage
\MTDInfoPage

\section*{Предисловие}
Физический практикум содержит описания лабораторных работ для учащихся 10-х классов лицея \textnumero1580 при МГТУ имени Н.~Э.~Баумана.

На выполнение каждой работы отводится два академических часа занятий. Подготовку к выполнению работ учащиеся производят в часы их самостоятельной работы.

Основной задачей лабораторных занятий является приобретение навыков в обращении с измерительными приборами, знакомство с простейшими приемами обработки результатов измерений и привитие учащимся навыков самостоятельной работы.

Наряду с этим выполнение лабораторных работ способствует более осознанному пониманию физических явлений и законов.

В описании работ даны краткая теория, методика выполнения работы, последовательность измерительных операций, а также простейшие приемы обработки результатов измерений.

\section*{Методические указания}
При подготовке к выполнению лабораторной работы нужно ознакомиться с ее содержанием, изучить по рекомендованной литературе теоретический материал, дать ответы на контрольные вопросы, продумать измерительные операции, оформить лабораторный журнал. В качестве лабораторного журнала используют общую тетрадь. Оформление каждой работы начинают с новой страницы.

В тетрадь вписывают название, номер работы, дату выполнения, цель, схемы установки, перечень приборов, таблицы, расчетные формулы.

Для вспомогательных записей и расчетов отводят четные страницы; схемы, таблицы выполняются в карандаше; все записи делают чернилами; графики вклеивают в тетрадь.

По окончании всех измерений рассчитывают искомые величины и их погрешность. В конце работы пишут заключение, в котором:

\begin{itemize}
  \item указывают, что и каким методом определили
  \item приводят окончательный результат измерений
  \item приводят краткое обсуждение полученного результата
\end{itemize}
\newpage

\setcounter{section}{0} % Введение
\section*{Введение}
\subsection{Предварительное знакомство с теорией погрешностей}
В физической лаборатории вы сможете непосредственно наблюдать те явления, которые будете изучать на лекциях и по учебнику, сможете познакомиться с наиболее важными современными приборами и методами измерений, освоить правила обработки и оформления результатов измерения.

 Физическая величина ---  это характеристика одного из свойств физического объекта, качественно общая для разных объектов, но присущая данному объекту в количественном отношении. Физическими величинами являются, например, масса, сила, температура.

  Для получения количественной характеристики физической величины ---  значения физической величины ---  устанавливают единицы физической величины (неверным является словосочетание <<единица измерения>>). Выбор единиц физических величин, определение некоторых из них как основных и формирование систем единиц физических величин осуществляются в процессе развития науки и техники и определяются удобством практического использования. В Российской Федерации принята как обязательная для использования Международная система единиц (СИ).

  % ИЗМ: словосочетание <<единица измерения>> -> словосочетание <<единица измерения>> некорректно (косноязычие)

  В основе лабораторного эксперимента лежит измерение ---  нахождение значения физической величины опытным путем с помощью специальных технических средств. Выделяют измерение прямое, при котором искомое значение величины находят непосредственно из опытных данных, и косвенное, при котором искомое значение величины находят на основании известной зависимости между этой величиной и величинами, подвергаемыми прямым измерениям.



Для производства измерения необходимо взаимодействие средств измерения с элементами измеряемой физической системы. Любое взаимодействие тел изменяет их состояние, а значит, в процессе любого измерения мы определяем значения физических величин, которое они принимают в процессе взаимодействия исследуемого тела с измерительными средствами. Это означает, что полученные в процессе измерения значения исследуемых физических величин отличаются от их значений в отсутствие измерительного взаимодействия. Следовательно, в задачу любого измерения входит не только нахождение значения самой величины, но и оценка допущенной при измерении погрешности.

Погрешности измерений подразделяются на систематические, случайные и грубые.

Систематическая составляющая погрешности измерения --- погрешность, обусловленная одной и той же причиной, которая может быть известна заранее или определена в процессе дополнительных исследований. Эта погрешность остаётся постоянной или закономерно изменяется при повторных измерениях одной и той же величины. Она обусловлена, в основном, методом измерений, игнорированием % ИЗМ: было "неучётом", такое слово есть?
некоторых постоянно действующих факторов и погрешностями измерительных приборов, внесёнными при их изготовлении и градуировке. Систематические погрешности в принципе можно устранить, учитывая их в виде поправок к показаниям приборов, выбирая более точный метод или прибор и т. п. На практике некоторая систематическая погрешность результата измерений существует всегда, поскольку точность измерительного прибора имеет конечное значение. При обработке результатов измерений оценивают диапазон значений измеряемой величины, обусловленный точностью прибора.

Случайная составляющая погрешности измерения ---  погрешность, изменяющаяся случайным образом при повторных измерениях одной и той же величины. Эти погрешности обусловлены разными причинами, которые заранее не известны. Такими причинами могут быть неконтролируемые изменения внешних условий (температуры, давления, влажности, вибраций, освещённости и т. п.). Случайные погрешности принципиально неустранимы, но существуют способы их уменьшения.

% ИЗМ: 1. "существуют" на "есть" 2. существуют на "можно выделить", стиль, тавтология
Можно выделить общие закономерности в появлении случайных погрешностей:
\begin{itemize}
  \item отклонения от истинного значения измеряемой величины в
сторону завышения и в сторону занижения встречаются в среднем
одинаково часто;
  \item малые отклонения встречаются чаще, чем большие.
\end{itemize}

Наличие таких закономерностей позволяет надеяться, что среднее значение результатов нескольких измерений близко к истинному и, чем больше число измерений, тем точнее полученный результат.

Грубая погрешность возникает как следствие ошибок измерений (небрежности при чтении показаний приборов, неправильного включения прибора и т. п.). Эта погрешность существенно превышает ожидаемую при данных условиях погрешность. Результаты, полученные с грубой погрешностью, следует исключать из рассмотрения (как говорят, <<отбрасывать>>). Методы математической статистики позволяют оценить ожидаемую погрешность и тем самым оценить наличие грубой погрешности. В принципе, систематические погрешности можно устранить. Исключить случайные погрешности отдельных измерений невозможно, хотя математическая теория случайных явлений позволяет уменьшить влияние этих погрешностей на окончательный результат измерений и установить разумное значение погрешностей. Для этого необходимо произвести не одно, а несколько измерений, причем, чем меньшее значение случайной погрешности мы хотим получить, тем большее число измерений нужно произвести.

Однако нет смысла производить измерений больше, чем это необходимо. Число измерений целесообразно выбирать таким, чтобы случайная погрешность была несколько меньше систематической, а последняя определяла окончательную погрешность результата.

\subsection {Абсолютная и относительная погрешность}
Абсолютной погрешностью измерения называют разность между
истинным и измеренным значениями величины. 

\begin{equation}
\label{eq:abs-error}
\Delta x = X_{\text{ист}} - x_i,
\end{equation}


\end{document}