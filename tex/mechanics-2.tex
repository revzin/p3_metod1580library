% -*-coding: utf-8 -*-
% Держать в начале каждого файла!

\documentclass[a4paper, 12pt]{extarticle}
\usepackage{metod}

\MTDSetPhysSection{Механика}
\MTDSetTitle{Изучение законов поступательного движения твердого тела}
\MTDDesignator{М--2}
\MTDSetGrade{10}

\MTDSetAuthors{И.~Н.~Грачева, В.~И.~Гребенкин, А.~Е.~Иванов, И.~А.~Коротова,
Е.~И.~Красавина, А.~В.~Кравцов, Н.~С.~Кулеба, Б.~В.~Падалкин,
Г.~Ю.~Шевцова, Т.~С.~Цвецинская}

\MTDSetEditorsGenCase{И. Н. Грачевой, А. Е. Иванова, А. В. Кравцова}

\begin{document}

\MTDTitlePage
\MTDInfoPage

\setcounter{section}{2}

\subsection{Цель работы}
Целью работы является экспериментальная проверка законов поступательного равноускоренного движения твердого тела. 

\subsection{Основные теоретические сведения}
Твердым телом в механике называется такое тело, у которого во время движения не изменяются расстояния между частицами, составляющими тело. 

Движение твердого тела называется поступательным, если прямая, проведенная через две любые точки тела, остается при движении параллельной самой себе.

При поступательном движении все частицы тела имеют одинаковые по величине и направлению скорости и ускорения. Поэтому движение тела можно рассматривать как движение одной материальной точки, имеющей массу всего тела. В качестве такой материальной точки обычно выбирают центр масс тела. 

В случае равноускоренного поступательного движения твердого тела (материальной точки) уравнение движения и уравнение для скорости имеют вид:

\begin{equation}
\label{eq:m2-uni-acc-motion}
\vec{r} = \vec{r_0} + \vec{v_0}t + \frac{\vec{a}t^2}{2} % \n хочу
\vec{v} = \vec{v_0} + \vec{a}t,
\end{equation}
где $\vec{r}$ "--- радиус-вектор конечного положения тела; \\% как сделать отступ как в методичке 
$\vec{r_0}$ "--- радиус-вектор начального положения тела; \\
$\vec{v_0}$ "--- начальная скорость тела; \\
$\vec{v}$ "--- скорость тела в момент времени $t$; \\
$\vec{a}$ "--- ускорение. 

Для прямолинейного равноускоренного движения с нулевой начальной скоростью пройденный путь и скорость зависят от времени движения как %мб двоеточие?

\begin{equation}
\label{eq:m2-uni-acc-lin-motion}
s = \frac{at^2}{2} \\
v = at 
\end{equation}

\subsection{Описание экспериментальной установки}
\begin{figure}[h] %как сделать сбоку?
\caption{Схема экспериментальной установки \label{fig:m2-atwood-machine}}
\end{figure}
Работа выполняется на машине Атвуда (рис.~\ref{fig:m2-atwood-machine}). 

Установка состоит из вертикальной стойки~1 с сантиметровыми делениями и блока~2, укрепленного в подшипнике. Через блок переброшена нить с грузами~6 одинаковой массы. Нить зажимается электромагнитом~3, электропитание которого включается и выключается с пульта электронного секундомера~7. Один из грузов~6 может проходить через кольцо~5, на котором установлен контакт <<Старт>> секундомера. На приемной площадке~4 установлены контакты <<Стоп>> секундомера. 



\end{document}