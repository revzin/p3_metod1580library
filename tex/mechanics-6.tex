% -*-coding: utf-8 -*-
% Держать в начале каждого файла!

\documentclass[a4paper, 12pt]{extarticle}
% 14 пт - жесть
\usepackage{metod}

\MTDSetPhysSection{Механика}
\MTDSetTitle{Изучение закона сохранения импульса при упругом соударении}
\MTDDesignator{М--6}
\MTDSetGrade{10}

\MTDSetAuthors{И.~Н.~Грачева, В.~И.~Гребенкин, А.~Е.~Иванов, И.~А.~Коротова,
Е.~И.~Красавина, А.~В.~Кравцов, Н.~С.~Кулеба, Б.~В.~Падалкин,
Г.~Ю.~Шевцова, Т.~С.~Цвецинская}

\MTDSetEditorsGenCase{И.~Н.~Грачевой, А.~Е.~Иванова, А.~В.~Кравцова}

\newcommand{\eps}{\epsilon}
\newcommand{\nisum}{\sum\limits_{i=1}^{i=n}} %WTF?!
\newcommand{\isum}{\sum\limits_{i=1}^{n}}
\newcommand{\issum}{\sum\limits_{i}}

\begin{document}
\MTDTitlePage
\MTDInfoPage

\setcounter{section}{6}

\subsection{Цель работы}
Целью работы является экспериментальная проверка условий равновесия тела с закрепленной осью вращения.

\subsection{Основные теоретические сведения}
Твердое тело с закрепленной неподвижной осью вращения остается в  покое, если сумма моментов сил, действующих на тело, относительно оси вращения равна нулю:

\[
\isum M_i = 0. % Точку поставил
\]

При этом сумма сил, действующих на тело, также равна нулю из-за возникновения сил реакций со стороны крепления оси вращения.

\begin{figure}
\caption{Схема установки \label{fig:m6-disk-device}}
\end{figure}



\subsection{Описание экспериментальной установки}

Установка изображена на рис.~\ref{fig:m6-disk-device}. В установке неоднородный диск~\emph{1} может поворачиваться вокруг горизонтальной оси, проходящей через его геометрический центр. Положение диска с помощью указателя определяется по круговой шкале, деления которой нанесены на раме~\emph{2}, укрепленной на стойке~\emph{3}. Боковыми винтами~\emph{8} фиксируют положение диска.  Установка всего прибора по отвесу производится регулировочными винтами~\emph{4} в плите~\emph{9}. Изменение высоты центра масс фиксируется по шкале вертикальной линейки~\emph{5}. Динамометр~\emph{6} шарнирно прикрепляется одним концом к раме, другим "--- к диску. При повороте диска пружина растягивается. Сила натяжения пружины измеряется по шкале динамометра. К диску на тонкой нити подвешиваются грузы~\emph{7}. Вес диска измеряется спаренным динамометром, не входящим в состав установки и расположенном на столе рядом с описанной  установкой.


\subsection{Порядок выполнения работы}

\begin{enumerate}
\item Во время домашней подготовки к  работе  выполните в лабораторном журнале таблицы~\ref{tab:m6-first} -- \ref{tab:m6-third}.
\item Ознакомьтесь с установкой и получите у преподавателя допуск к выполнению работы.
\end{enumerate}

\MTDTask{Определение массы и положения центра тяжести неоднородного диска}
\begin{enumerate}
\item
С помощью спаренного динамометра, укрепленного на штативе,
взвесьте три раза диск. Измеренные значения силы тяжести~$P_i$ запишите в таблицу~\ref{tab:m6-first}. % ИЗМ Было: Пользуясь разделом В.4 настоящего сборника,
Пользуясь разделом~В.4 вводной лабораторной работы, определите погрешность измерения~$P$.


\item
Найдите положение центра тяжести диска следующим образом: % Была точка
Прикрепив к диску лист миллиметровой бумаги с вырезом посередине (вырез должен быть сделан так, чтобы пересечение двух взаимно перпендикулярных утолщенных линий <<миллиметровки>> проходило через центр диска), подвесьте диск вместе с отвесом последовательно в трех точках диска. На пересечении трех прямых, проведенных по отвесу, отметьте точкой положение центра тяжести. Если на пересечении прямых образуется маленький треугольник, то точку центра тяжести ставят в точке пересечения медиан этого треугольника.

\end{enumerate}


\end{document} 